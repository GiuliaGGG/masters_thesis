\documentclass{article}
\usepackage{graphicx} % Required for inserting images
\usepackage{xcolor}
\usepackage[style=apa, backend=biber]{biblatex}

\addbibresource{references.bib}

\title{Master Thesis First Draft}
\author{Giulia Maria Petrilli}
\date{2025}

\begin{document}
\maketitle
\begin{abstract}

\end{abstract}


\section{Introduction}

Boycott, Divestment, and Sanctions (BDS) is a nonviolent Palestinian-led movement promoting boycotts, divestments, and economic sanctions against Israel. 
It does so by promoting various forms of boycotts against companies that are perceived to be complicit in the Israeli occupation of Palestinian territories.
The movement aims to pressure these companies to change their policies and practices regarding the Israeli occupation of Palestinian territories.
BDS has been active since 2005, but with the recent genocide in Gaza, the movement has gained significant traction and visibility worldwide. 
There are several articles and statements published that claim the effectiveness of boycotts on company performance, 
but while a correlation seems to be apparent in many cases, causation is delicate, and the current literature lacks a robust empirical analysis of the impact of boycotts on company performance. 
This thesis provides a methodology to achieve such goal, using a Synthetic Control Method to formulate a credible counterfactual for McDonald's performance in the absence of a boycott.

The current literature on the impact of boycotts measures company performance using share or stock prices \cite{pruitt1986determining}, \cite{eva2025impact}.
While analyzing stock market price patterns for boycotted companies gives a good estimate on the trust the investors place in them, it does not accurately capture consumer response, and the consequent impact on revenue. The Boycott, Divest, Sanction (BDS) movement has been deploying their advocacy effort into companies that are targeted for boycotts.



\section{Literature Review}
Before delving into the project, it is necessary to pan out the theory of boycotts this theoriy is rooted in, the methodology used and the gap your study fills
In late September 1880, close to a hundred farmers and peasants assembled at the estate of the local landlord Charles Cunningham Boycott in County Mayo, Ireland. They were protesting against the rent increases imposed by boycotts, and started to isolate the man and his family from business operations. 
The practice of organized non-consumption, now named 'boycott', has spread through the years, becoming a common tool for social movements to express their discontent towards companies or countries. The boycott, or threat of boycott, has been proven to have an effect when firms see a dent in their performance andadjust their behavious to comply to the consumer's preference, whose purchases depend on. 
The most notable case of boycott in recent history is the South African anti-apartheid movement, which used boycotts as a key strategy to pressure the South African government. Differently, the current boycott is bottom up, meaning that there is not a state ban on certain companies, but the phenomenon is rather grassroot. BDS mainly operates on social media. This makes a causal fit rather interesting, because wether in the south african case one can quite clearly connect th drop in sales to the ban, in this case wether one boycotts is fully subject fully to the consumer. 
This thesis is placed in the corpus of economic boycott. A paper has done this analysis but using share prices, which measures trust in a company and not if people buy it or not

\section{Methodology}
% Add \usepackage{xcolor} to the preamble
This thesis uses the Synthetic Control Method (SCM) to estimate the impact of the BDS boycott calls on McDonald's \textcolor{red}{net income margin}. The SCM is a data-driven procedure that constructs a synthetic version of the treated unit by weighting a combination of control units (other fast-food companies not targeted by BDS). This synthetic control serves as a counterfactual, representing what would have happened to McDonald's \textcolor{red}{net income margin} in the absence of the boycott. This method is particularly suitable to this analysis because  
A salient characteristic of the SCM is its ability to account for unobserved confounders that vary over time, making it particularly suitable for this analysis. Since SCM does not use randomization, the assignment process is critical to ensure the validity of the results. In the context of this thesis, the assignment process involves picking a poll of control units that could have been a boycott target, but were not. 
the targeted companies and rationale (\textcite{bdsStrategic2024}). The main points to select treatment are level of complicity, potential for cross-movement coalition-building, media appeal, and potential for success. This thesis uses the BDFS criteria to select both the taret company and the control group. 

\subsection{Data}
SEC unaudited trimestral data, SEC edgar. I used Income Statement (Statement of Operations)
The code to retrieve this finanical filings were provided by a github repo.

\subsection{Causal Fit}
\section{Preliminary results }
\section{Discussion}

\printbibliography
\end{document}
