\documentclass{article}
\usepackage{graphicx} % Required for inserting images
\usepackage{xcolor}
\usepackage[style=apa, backend=biber]{biblatex}

\addbibresource{references.bib}

\title{Master Thesis First Draft}
\author{Giulia Maria Petrilli}
\date{2025}

\begin{document}
\maketitle
\begin{abstract}

\end{abstract}


\section{Introduction}

Boycott, Divestment, and Sanctions (BDS) is a nonviolent Palestinian-led movement promoting boycotts, divestments, and economic sanctions against Israel. 
It does so by promoting various forms of boycotts against companies that are perceived to be complicit in the Israeli occupation of Palestinian territories.
The movement aims to pressure these companies to change their policies and practices regarding the Israeli occupation of Palestinian territories.
BDS has been active since 2005, but with the recent genocide in Gaza, the movement has gained significant traction and visibility worldwide. 
There are several articles and statements published that claim the effectiveness of boycotts on company performance, 
but while a correlation seems to be apparent in many cases, causation is delicate, and the current literature lacks a robust empirical analysis of the impact of boycotts on company performance. 
This thesis provides a methodology to achieve such goal, using a Synthetic Control Method to formulate a credible counterfactual for McDonald's performance in the absence of a boycott.

The current literature on the impact of boycotts measures company performance using share or stock prices \cite{pruitt1986determining}, \cite{eva2025impact}.
While analyzing stock market price patterns for boycotted companies gives a good estimate on the trust the investors place in them, it does not accurately capture consumer response, and the consequent impact on revenue. The Boycott, Divest, Sanction (BDS) movement has been deploying their advocacy effort into companies that are targeted for boycotts.



\section{Literature Review}
In late September 1880, close to a hundred farmers and peasants assembled at the estate of the local landlord Charles Cunningham Boycott in County Mayo, Ireland. They were protesting against the rent increases imposed by boycotts, and started to isolate the man and his family from business operations. 
The practice of organized non-consumption, now named 'boycott', has spread through the years, becoming a common tool for social movements to express their discontent towards companies or countries. 
The most notable case of boycott in recent history is the South African anti-apartheid movement, which used boycotts as a key strategy to pressure the South African government. Differently, the current boycott is bottom up, meaning that there is not a state ban on certain companies, but the phenomenon is rather grassroot. BDS mainly operates on social media. This makes a causal fit rather interesting, because wether in the south african case one can quite clearly connect th drop in sales to the ban, in this case wether one boycotts is fully subject fully to the consumer. 

\section{Methodology}
\subsection{Data}
Unaudited trimestral data, SEC edgar. I used Income Statement (Statement of Operations)

The assignment process is 
BDS focuses on boycotting a small number of companies to achieve a maximum impact, hence,

\subsection{Causal Fit}
\section{Preliminary results }

\printbibliography
\end{document}
